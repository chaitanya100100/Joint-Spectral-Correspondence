\documentclass{beamer}
%
% Choose how your presentation looks.
%
% For more themes, color themes and font themes, see:
% http://deic.uab.es/~iblanes/beamer_gallery/index_by_theme.html
%
\mode<presentation>
{
  \usetheme{default}      % or try Darmstadt, Madrid, Warsaw, ...
  \usecolortheme{default} % or try albatross, beaver, crane, ...
  \usefonttheme{default}  % or try serif, structurebold, ...
  \setbeamertemplate{navigation symbols}{}
  \setbeamertemplate{caption}[numbered]
} 

\usepackage[english]{babel}
\usepackage[utf8x]{inputenc}

\title[Joint Spectral Correspondence for Disparate Image Matching]{Joint Spectral Correspondence for Disparate Image Matching \\ . \\ \large Matlab Implementation Details}
\author{Chaitanya Patel}
\institute{CVIT, IIIT-Hyderabad}
\date{June, 2017}

\begin{document}

%-------------------------
% Title Page
%-------------------------
\begin{frame}
  \titlepage
\end{frame}

%-------------------------
% Index
%-------------------------
\begin{frame}{Outline}
  \tableofcontents
\end{frame}


%-------------------------
% Getting dense sift features
%-------------------------
\section{Dense SIFT features}
\begin{frame}{Dense SIFT features}
\begin{itemize}
\item Each image is resized to control total number of pixels
\item One way is to use \textbf{\texttt{vl\_dsift}} function. It will give 128D feature at each keypoint.      	  \begin{itemize}
	\item binsize 6 pixels 
   	\item step(stride size) 4 pixels
\end{itemize}
\item According to paper, sift feature is extracted for each key point for two scales i.e. two binsize : 10 and 6 pixels and concatenate them to create 256D feature at each key point
\item For that, \textbf{\texttt{vl\_sift}} is called twice for each binsize with keypoints specified in frames
\end{itemize}
\end{frame}

%-------------------------
% Adjacency Matrix
%-------------------------
\section{Adjacency Matrix}
\begin{frame}{Adjacency Matrix}
\begin{itemize}

\item \textbf{\texttt{pdist}} is used to create adjacency matrix for intra image pixels
\item \textbf{\texttt{pdist2}} is used to create adjacency matrix for inter image pixels
\item cosine distance is used
\item They are concatenated to create joint image graph adjacency matrix

\end{itemize}
\end{frame}


%-------------------------
% Laplacian and its eigen decomposition
%-------------------------
\section{Laplacian and its Eigen decomposition}
\begin{frame}{Laplacian and its eigen decomposition}
\begin{itemize}

\item Degree Matrix is obtained from Adjacency Matrix
\item Normalized Laplacian is calculated using formula $L = I - DWD$
\item \textbf{\texttt{eigs}} is used with parameter \textbf{\texttt{'sm'}} to get 5-6 eigenvectors with smallest eigenvalues

\end{itemize}
\end{frame}


%-------------------------
% Reconstruction using Eigenvectors
%-------------------------
\section{Reconstruction using Eigenvectors}
\begin{frame}{Reconstruction using Eigenvectors}
\begin{itemize}

\item Eigenvector is divided into two halves - one for each image
\item values are put back to each key point and other values are linearly interpolated using \textbf{\texttt{interp2}}

\end{itemize}
\end{frame}


%-------------------------
% MSER features extraction and matching
%-------------------------
\section{MSER features extraction and matching}
\begin{frame}{MSER features extraction and matching}
\begin{itemize}
	\item \textbf{\texttt{vl\_mser}} is used for feature extraction
    \item \textbf{\texttt{knnsearch}} is used for feature matching in both images
\end{itemize}
\end{frame}


\end{document}
